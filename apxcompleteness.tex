\documentclass[]{article}

% Package `hyperref` must come before package `complexity`.
\usepackage[pdftitle={APX completeness}, pdfauthor={Jeffrey Finkelstein}]{hyperref}

\usepackage{amsmath}
\usepackage{amssymb}
\usepackage{amsthm}
\usepackage{complexity}

\theoremstyle{plain}
%\newtheorem{conjecture}{Conjecture}
%\newtheorem{proposition}{Proposition}
\newtheorem{theorem}{Theorem}
%\newtheorem{todo}{TODO}

%\theoremstyle{definition}
%\newtheorem{definition}{Definition}
%\newtheorem{openquestion}{Open question}

\mathchardef\mhyphen="2D

%\newenvironment{justification}{\begin{proof}[Justification]}{\end{proof}}

%\newcommand{\definitionautorefname}{Definition}

\newcommand{\APr}{\leq_{AP}^{P}}
\newcommand{\email}[1]{\href{mailto:#1}{\nolinkurl{#1}}}
%\newcommand{\lb}{\left\{}
%\newcommand{\pb}{\textsf{pb}}
%\newcommand{\rb}{\right\}}
%\newcommand{\st}{\,\middle|\,}

\author{Jef{}frey Finkelstein}
\date{\today}
\title{\texorpdfstring{\textsc{Maximum 3-Satisfiability}}{Maximum 3-Satisfiability}}

\begin{document}

\maketitle

\begin{theorem}[{\cite[Theorem~3.7]{book}}]
  Let $Q$ be an \NP-complete decision problem and $P$ be a minimization problem in \NPO{} (so $P=(I, S, m, \min)$).
  Suppose there exist two polynomial time computable functions $f\colon\Sigma^*\to I$ and $c\colon \Sigma^*\to\mathbb{N}$ and an $\epsilon\in\mathbb{R}^+$ with $\epsilon>0$ such that for any $x\in\Sigma^*$,
  \begin{enumerate}
  \item if $x \in Q$ then $m^*(f(x)) = c(x)$, and
  \item if $x \notin Q$ then $m^*(f(x)) = (1 + \epsilon) \cdot c(x)$.
  \end{enumerate}
  Then for any $r < 1 + \epsilon$, no polynomial time $r$-approximation algorithm for $P$ exists unless $\P=\NP$.
\end{theorem}
\begin{proof}
  In order to produce a contradiction, suppose that there exists a polynomial time $r$-approximator $A$ for $P$ with $r < 1 + \epsilon$.
  Define algorithm $A'$ as follows: $A'(x)$ accepts if and only if $m(f(x), A(x)) < (1 + \epsilon) \cdot c(x)$, for all $x\in\Sigma^*$.
  We will show that $A'$ is a polynomial time algorithm for the \NP-complete problem $Q$, and hence $\P=\NP$.

  Since $f$, $A$, $c$, and basic arithmetic operations such as addition and multiplication are polynomial time computable functions, so is $A'$.
  In order to show that $A'$ is correct we must consider two cases.
  \begin{enumerate}
  \item
    Suppose $x \in Q$.
    Since $A$ is an $r$-approximator and $r < 1 + \epsilon$, we have
    \begin{displaymath}
      \frac{m(f(x), A(f(x)))}{m^*(f(x))} \leq r < 1 + \epsilon.
    \end{displaymath}
    Since $m^*(f(x)) = c(x)$ by hypothesis, this implies
    \begin{align*}
      m(f(x), A(f(x))) & < (1 + \epsilon) \cdot m^*(f(x)) \\
      & = (1 + \epsilon) \cdot c(x).
    \end{align*}
    Hence $A'$ will accept on input $x$.
  \item
    If $x \notin Q$ then $m^*(f(x)) = (1 + \epsilon) \cdot c(x)$ by hypothesis.
    Since $P$ is a minimization problem, $m(f(x), A(f(x))) \geq m^*(f(x))$.
    Combining the two inequalities, we find $m(f(x), A(f(x))) \geq (1 + \epsilon) \cdot c(x)$.
    Hence, $A'$ will reject on input $x$.
  \end{enumerate}
  We have shown that $A'$ is a correct polynomial-time computable algorithm which decides the $Q$, and the conclusion follows.
\end{proof}

\begin{theorem}[{\cite[Theorem~6.3]{book}}]
  $\textsc{Maximum 3-Satisfiability} \notin \PTAS$ unless $\P=\NP$.
\end{theorem}
\begin{proof}
  
\end{proof}

\begin{theorem}
  \textsc{Maximum 3-Satisfiability} is complete for \APX{} under $\APr$ reductions.
\end{theorem}

We first present the idea behind this proof.
Recall the Cook-Levin technique for transforming a Turing machine, whose running time is bounded by a polynomial in the length of its input, into a Boolean circuit such that for any input, the Turing machine accepts if and only if the Boolean circuit outputs 1.
We use this technique to transform a Turing machine which decides an arbitrary language in \APX{} into a Boolean formula representing the Boolean circuit.

\begin{proof}
  We show a generic reduction from 
\end{proof}

\section*{About this work}

Copyright 2012 Jef{}frey Finkelstein.

This work is licensed under the Creative Commons Attribution-ShareAlike License 3.0.
Visit \mbox{\url{https://creativecommons.org/licenses/by-sa/3.0/}} to view a copy of this license.

The \LaTeX{} markup which generated this document is available on the World Wide Web at \mbox{\url{https://github.com/jfinkels/apxcompleteness}}.
It is also licensed under the Creative Commons Attribution-ShareAlike License.

The author can be contacted via email at \email{jeffreyf@bu.edu}.

\bibliographystyle{plain}
\bibliography{references}

\end{document}
